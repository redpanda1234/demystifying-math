\documentclass[10pt]{article}
\usepackage[tmargin=1.25in,lmargin=1in,rmargin=1in,bmargin=1in,paper=letterpaper]{geometry}
\usepackage{amsmath,amssymb,amsthm}
\usepackage{multirow,color}
\usepackage{fancyhdr,ifthen,lastpage}
\usepackage{relsize,changepage}
\usepackage{enumitem,empheq,float,caption}
\usepackage{tikz,tikz-qtree,tikz-qtree-compat}
\usetikzlibrary{shapes,backgrounds,calc,patterns,snakes,angles,quotes}
\usepackage[hang,flushmargin]{footmisc}
\usepackage[hidelinks]{hyperref}
\usepackage{framed}
\usepackage{skak}
\usepackage[normalem]{ulem}
\usepackage{textcomp}
\usepackage{pifont}
\newcommand{\cmark}{\ding{51}}%
\newcommand{\xmark}{\ding{55}}
\usepackage{fix-cm}
\usepackage{titling}
\usepackage[yyyymmdd,hhmmss]{datetime}
\usepackage{parskip}
% --------------------------------------------------------------------
\renewenvironment{leftbar}[1][\hsize]
{%
    \def\FrameCommand
    {%
        {\color{black}\vrule width 1.5pt}%
        \hspace{7pt}%%
        \fboxsep=\FrameSep%
    }%
    \MakeFramed{\hsize#1\advance\hsize-\width\FrameRestore}%
}%
{\endMakeFramed}
\newenvironment{leftbara}[1][\hsize]
{%
    \def\FrameCommand
    {%
        {\color{gray}\vrule width 1.5pt}%
        \hspace{7pt}%
        \fboxsep=\FrameSep%
    }%
    \MakeFramed{\hsize#1\advance\hsize-\width\FrameRestore}%
}%
{\endMakeFramed}
\newcommand\irregularcircle[2]{% radius, irregularity
  let \n1 = {(#1)+rand*(#2)} in
  +(0:\n1)
  \foreach \a in {10,20,...,350}{
    let \n1 = {(#1)+rand*(#2)} in
    -- +(\a:\n1)
  } -- cycle
}
\newcommand{\cvdots}[1][=]{\mathrel{\makebox[\widthof{#1}]{\vdots}}}
\DeclareMathOperator*{\bigplus}{\text{\fontsize{25}{35}\selectfont $+$}}
\DeclareMathOperator*{\bigtimes}{\text{\fontsize{25}{35}\selectfont $\times$}}
% --------------------------------------------------------------------
\newtheorem*{remark}{Remark}
\theoremstyle{definition}
\newtheorem{definitiona}{Definition}[section]
\newtheorem{sketcha}{Sketch}[section]
\newtheorem{techniquea}{Technique}
\newtheorem{example}{Example}[section]
\renewcommand\qedsymbol{$\blacksquare$}
\let\olddef\definition
\newenvironment{definition}{\begin{leftbar}[.95\linewidth]\begin{definitiona}}{\end{definitiona}\end{leftbar}}
\newenvironment{sketch}{\begin{leftbara}[.95\linewidth]\begin{sketcha}}{\end{sketcha}\end{leftbara}}
\newenvironment{technique}{\begin{leftbara}[.95\linewidth]\begin{techniquea}}{\end{techniquea}\end{leftbara}}
\newenvironment{law}{\begin{leftbar}[.95\linewidth]\begin{definitiona}}{\end{definitiona}\end{leftbar}}
\usepackage{cancel}

\newcommand{\abs}[1]{\left| #1 \right|}

\let\oldref\ref
\renewcommand{\ref}[1]{(\oldref{#1})}

\begin{document}
\pagestyle{fancy}

\setlength{\headheight}{0pt}
\setlength\parindent{0pt}

\lhead{ Punahou AP Physics C -- Spr. 2016}
\rhead{Forest Kobayashi}
\chead{\bf Electricty and Magnetism}

\setlength{\headheight}{0pt}
\pretitle{\begin{center}\Huge\bfseries}
\posttitle{\par\end{center}\vskip 0.5em}
\preauthor{\begin{center}\Large\ttfamily}
\postauthor{\end{center}}
\predate{\par\large\centering}
\postdate{\par}

\title{Electricty and Magnetism}
\author{Forest Kobayashi}
\date{Last modified: \today, at \currenttime}
% {\Huge \centering \textbf{A brief note to all students } }
\maketitle
\thispagestyle{empty}
\clearpage
\vspace{5cm}
\begin{center}
\Large\bfseries Dedication:
\end{center}
% \begin{center}
{\large%
% \emph{
%This text is dedicated to any student who, at some point or another,
%has sat alone in their room (possibly crying), asking themself `Why
%can't I be smart?', 'Why can't I be good at Math?', `Why is it so
%easy  for everybody but me?', or any other analogous question.  \\~\\
%I have too---many, many, \emph{many} times.  You are not alone. }
% }
% \end{center}
\thispagestyle{empty}
\clearpage
\tableofcontents
% ========================================================================
\clearpage
\pagestyle{fancy}
\section{Preface}
This document is (with some fairly minor changes) essentailly a
glorified write-up of the work I completed for my high school's AP
Physics C course.  The course focused on inquiry-based learning, and
much of the educational value derived from it depended on that fact.
As such, I would highly advise that any student reading this document
follow along with a pencil and paper, and attempt to work out the
solutions before looking at the solution method.  I'll give my answer
at the end of the section, and my overall solution method at some
point later in the document.

One piece of advice before we launch into it: \textbf{don't be
  intimidated by algebra.}  Algebra is not where learning happens.(TOfinish)
\section{Electrostaics: Forces, Fields, and Gauss' Law}
\subsection{Electricity and Magnetism}
Electricity and Magnetism (commonly referred to simply as `E\&M') is a
phenomenally cool subject.  Oftentimes, students are divided into two
main `camps' when it comes to feelings with respect to E\&M: about
half tend to find E\&M very challenging, as the systems are more
abstract (it is in some sense `harder' to imagine the minutia of
interactions between electrons than it is, say, to imagine a ball
rolling down a ramp).  The other half tend to find E\&M to be far
\emph{simpler} than mechanics, as the systems we typically analyze are
far more constrained, and generally `cleaner' than those we encounter
in mechanics.

The goal of this text is to move all its readers from the former camp
to the latter.
\subsection{Coulomb's Law}
We'll begin our discussion of E\&M by talking about interactions
between \textbf{charges}. The following is (very lightly) paraphrased
directly from my high school packets.

In 1747, Benjamin Franklin became the first person to hypothesize that
electric forces were due to a property of matter called \emph{electric
  charge}, which exists in two varieties: positive and negative.
Opposite charges attract each other, and like charges repel, with a
force described by \textbf{Coulomb's Law}.  Coulomb's Law is only
valid if we are dealing with `point charges'---i.e., charges that
don't have any dimension to them (meaning they are infinitely
small, and therefore have no width, height, or depth, existing only as
a singular point in space).  This should raise some eyebrows---how can
a physical object have no length, width, or depth?  Well, in terms of
real, physical systems, we won't see anything of that sort in this
text.  But, we will interact with a lot of objects that we can
\emph{show} (with some clever math) can be \emph{treated} as point
charges, without `losing' any descriptive information about our
system.  As such, understanding how to deal with systems involving the
interactions between point charges is important to being able to
describe more complex systems.

if $q_1$ and $q_2$ are two
point charges and $r$ is the distance between them, then they attract
or repel (according to their signs) with a force
\[F = \frac{k\abs{q_1}\abs{q_2}}{r^2}\]
\section{Electric Potential Energy, and Electric Potential}
\section{Capacitance and Capacitors}
\section{Current and Resistors}
\section{Electric Circuits}
\section{Magnetic Fields}
\section{Ampere's Law}
\section{Magnetic Induction and Faraday's Law}
\end{document}
