\documentclass[cm, 10pt, titlepage, oneside]{book}
% =============================================================== %
%                                                                 %
%                             Packages                            %
%                                                                 %
% =============================================================== %

% Making the page dimensions reasonable
\usepackage[margin=1in, tmargin=1.25in]{geometry}

% For most of the math symbols, and nice things like align
% environment
\usepackage{amsmath, amssymb}

% Fancy page header
\usepackage{fancyhdr}

% For customizing the enumerate environment
\usepackage{enumitem}

% For leftbar environment (for proving lemmas, if needed)
\usepackage{framed}

% For leftbar color
\usepackage{color}

% In case you want any diagrams
\usepackage{tikz}
\usetikzlibrary{arrows}
\usetikzlibrary{arrows.meta}
\usepackage{pgfplots}
\usepackage{asymptote}

% ...Which you would probably place in figure environments
\usepackage{float}

% Because I've learned my lesson about manual line breaks
\usepackage{parskip}

% For checkmark
\usepackage{pifont}

% --------------------- Hacky Control Seqs ---------------------- %
% For referencing labeled equations w/ number in parens
\let\oldref\ref
\renewcommand{\ref}[1]{(\oldref{#1})}

% Useful shortcut commands for common sets
\newcommand\CC{{\mathbb C}}
\newcommand\RR{{\mathbb R}}
\newcommand\QQ{{\mathbb Q}}
\newcommand\ZZ{{\mathbb Z}}
\newcommand\NN{{\mathbb N}}

% Commands to avoid exiting math mode
\newcommand{\st}{\text{ st }}
\newcommand{\lub}{\text{lub}}
\newcommand{\glb}{\text{glb}}

% Useful commands for paired delimiters that'll adjust to fit whatever
% argument you pass --> e.g. (<something>), [<something else>]
\newcommand{\set}[1]{\ensuremath{ \left\{ #1 \right\} }}
\newcommand{\pn}[1]{\left( #1 \right)}
\newcommand{\abs}[1]{\left| #1 \right|}
\newcommand{\bk}[1]{\left[ #1 \right]}
\newcommand{\vc}[1]{\left\langle #1 \right\rangle}

% I use this environment for proving any lemmas I use
\renewenvironment{leftbar}[1][\hsize]
{%
    \def\FrameCommand
    {%
        {\color{black}\vrule width 1.5pt}%
        \hspace{7pt}%
        \fboxsep=\FrameSep%
    }%
    \MakeFramed{\hsize#1\advance\hsize-\width\FrameRestore}%
}%
{\endMakeFramed}

% ------------------- Chapter Heading Formatting ------------------- %
\usepackage{titlesec}

\titleformat{\chapter}[display]{
  \normalfont\huge\bfseries}{
  \chaptertitlename\ \thechapter}{
  20pt}{
  \Huge}
\titlespacing*{\chapter}{0pt}{0pt}{10pt}

% And shorten the vspace after section headers. How to read?
%
%                {12pt plus 4pt minus 2pt}
%
%  - 12pt is what we would like the spacing to be
%  - plus 4pt means that TeX can stretch it by at most 4pt
%  - minus 2pt means that TeX can shrink it by at most 2pt

\titlespacing\section{0pt}{
  12pt plus 4pt minus 2pt}{
  0pt plus 2pt minus 2pt}


% ------------------------ Title Formatting ------------------------ %
\title{\bf Linear Algebra!}
\date{December 2017}
\author{Forest Kobayashi}

% Finally, enumerate formatting
\setenumerate[0]{label=(\alph*)}

% ================================================================== %
%                                                                    %
%                               Document                             %
%                                                                    %
% ================================================================== %

\begin{document}

% ----------------------- Make the Title page ----------------------- %

\frontmatter
\maketitle
\tableofcontents
\mainmatter

\chapter{Introduction}
  \section{What is Linear Algebra?}
    If you were to ask the average college student this question,
    they'd probably shudder, mumble something about boxes that you
    stick numbers in and do things to, then run away, screaming
    something about Linear Algebra being a scary \& traumatic time.
    And who can blame them? Often, it's presented as a subject whose
    only discernable features are large volumes of mind-numbing
    arithmetic, tangled messes of theorems, and gratuitously-involved
    formulae to memorize. Little attention is typically given to the
    rich geometric intuition that underlies its core concepts. This is
    made all the more egregious by the fact that key results are often
    presented \emph{without proof.} The horror, the horror!

    I think this is a real shame. Linear Algebra is a pretty nifty
    subject, and (in my opinion) can be much more intuitive than
    something like Calculus. For the purposes of this document, you
    can think of Linear Algebra as the \emph{mathematics of spaces}.
    In particular, Linear Algebra is all about abstracting the
    properties of some system into abstract \emph{positions} in an
    appropriate space. This allows us to reduce complex,
    possibly-abstract systems to something visual and tangible. For
    instance, solutions to various systems of equations can be
    represented as \emph{points} in a \emph{solution space}. This has
    further applications in fields like Quantum Mechanics, we can use
    abstract vector spaces to represent the ``state'' of a system, and
    treat observable quantities such as momentum as \emph{operators}
    that take our initial state and map it to some other.

    Basically, Linear Algebra is pretty cool. At least, I think so.
    I hope you will too after reading this!

  \section{How to Read this Document}
    If this is still unwritten, and you'd like it to be uh... not
    unwritten, bother me on github, messenger, or some other platform
    for grievance voicing.

\chapter{Euclidean Spaces}
  \section{Some Terms}
  \section{Vector Arithmetic}
    \subsection{Vector Addition}
    \subsection{Vector Multiplication}
      \subsubsection{The Dot Product}
      \subsubsection{The Cross Product}
  \section{Linear Independence}
  \section{Spaces}
    \subsection{Bases and Dimension}
    \subsection{Subspaces}

\chapter{Matrix Theory}
  \section{Matrices as Changes of Bases}
  \section{Matrix Arithmetic}
    \subsection{Scalars and Matrices}
    \subsection{Vectors and Matrices}
    \subsection{Matrices and Matrices}
  \section{The Determinant}
  \section{Eigenvectors and Eigenvalues}

\chapter{Abstract Vector Spaces}
  \section{Definition}
    \subsection{Fields}
    \subsection{The Vector Space Axioms}
  \section{Linear Transformations}
    \subsection{Matrices as Linear Transformations}
  \section{Inner Product Spaces}



\end{document}
